\addcontentsline{toc}{section}{INTRODUCCIÓN}
\section*{INTRODUCCIÓN}
En este capítulo se presenta y señala el origen, la problemática a resolver, los objetivos, los alcances, las limitaciones.\\

Se inicia con una contextualización a partir de una descripción global del área relacionada, luego se entra en detalle a la temática a abordar, finalmente se dan detalles de trabajo que se realizará y el significado que el estudio tiene en el avance del campo respectivo y su aplicación en el área investigativa.
\addcontentsline{toc}{subsection}{PLANTEAMIENTO DEL PROBLEMA}
\subsection*{\hspace{5mm}PLANTEAMIENTO DEL PROBLEMA}
Enfocarse en el problema que va a resolver en su trabajo, sin mencionar alternativas de solución. Utilizar referencias confiables que respalden sus afirmaciones.
\subsection*{\hspace{5mm}OBJETIVOS DEL ESTUDIO}
Cada objetivo de responder a dos interrogantes:\\
¿Qué voy a hacer? y ¿Para qué lo voy a hacer?\\
Inician con un verbo en infinitivo.
\addcontentsline{toc}{subsubsection}{Objetivo general}
\subsubsection*{\hspace{5mm}Objetivo general}
Expresa lo que el autor pretende alcanzar con su trabajo de investigación. Debe guardar estrecha relación con el problema a resolver.
\addcontentsline{toc}{subsubsection}{Objetivos específicos}
\subsubsection*{\hspace{5mm}Objetivos específicos}
El cumplimiento de estos objetivos converge en el cumplimiento del objetivo general.
\addcontentsline{toc}{subsection}{JUSTIFICACIÓN}
\subsection*{\hspace{5mm}JUSTIFICACIÓN}
¿Por qué el problema mencionado previamente debe ser solucionado? Mencionar los beneficios que trae su proyecto de forma directa o indirecta a la población local, nacional. Mostrar los beneficios para la academia, como aporta su trabajo en el área respectiva. Mostrar beneficios directos e indirectos para la industria. Apoyar sus afirmaciones mediante datos publicados por otros autores.
\addcontentsline{toc}{subsection}{ALCANCES Y LIMITACIONES}
\subsection*{\hspace{5mm}ALCANCES Y LIMITACIONES}
\addcontentsline{toc}{subsubsection}{Alcances}
\subsubsection*{\hspace{5mm}Alcances}
Condiciones de operación del sistema o solución propuesta.
\addcontentsline{toc}{subsubsection}{Limitaciones}
\subsubsection*{\hspace{5mm}Limitaciones}
Incluir todas las limitaciones físicas y rangos de operación del sistema propuesto. Si se trata de un algoritmo, indicar las condiciones sobre las cuales puede operar sin inconvenientes.
\newpage
